% Options for packages loaded elsewhere
\PassOptionsToPackage{unicode}{hyperref}
\PassOptionsToPackage{hyphens}{url}
\PassOptionsToPackage{dvipsnames,svgnames,x11names}{xcolor}
%
\documentclass[
  letterpaper,
  DIV=11,
  numbers=noendperiod]{scrreprt}

\usepackage{amsmath,amssymb}
\usepackage{iftex}
\ifPDFTeX
  \usepackage[T1]{fontenc}
  \usepackage[utf8]{inputenc}
  \usepackage{textcomp} % provide euro and other symbols
\else % if luatex or xetex
  \usepackage{unicode-math}
  \defaultfontfeatures{Scale=MatchLowercase}
  \defaultfontfeatures[\rmfamily]{Ligatures=TeX,Scale=1}
\fi
\usepackage{lmodern}
\ifPDFTeX\else  
    % xetex/luatex font selection
\fi
% Use upquote if available, for straight quotes in verbatim environments
\IfFileExists{upquote.sty}{\usepackage{upquote}}{}
\IfFileExists{microtype.sty}{% use microtype if available
  \usepackage[]{microtype}
  \UseMicrotypeSet[protrusion]{basicmath} % disable protrusion for tt fonts
}{}
\makeatletter
\@ifundefined{KOMAClassName}{% if non-KOMA class
  \IfFileExists{parskip.sty}{%
    \usepackage{parskip}
  }{% else
    \setlength{\parindent}{0pt}
    \setlength{\parskip}{6pt plus 2pt minus 1pt}}
}{% if KOMA class
  \KOMAoptions{parskip=half}}
\makeatother
\usepackage{xcolor}
\setlength{\emergencystretch}{3em} % prevent overfull lines
\setcounter{secnumdepth}{5}
% Make \paragraph and \subparagraph free-standing
\makeatletter
\ifx\paragraph\undefined\else
  \let\oldparagraph\paragraph
  \renewcommand{\paragraph}{
    \@ifstar
      \xxxParagraphStar
      \xxxParagraphNoStar
  }
  \newcommand{\xxxParagraphStar}[1]{\oldparagraph*{#1}\mbox{}}
  \newcommand{\xxxParagraphNoStar}[1]{\oldparagraph{#1}\mbox{}}
\fi
\ifx\subparagraph\undefined\else
  \let\oldsubparagraph\subparagraph
  \renewcommand{\subparagraph}{
    \@ifstar
      \xxxSubParagraphStar
      \xxxSubParagraphNoStar
  }
  \newcommand{\xxxSubParagraphStar}[1]{\oldsubparagraph*{#1}\mbox{}}
  \newcommand{\xxxSubParagraphNoStar}[1]{\oldsubparagraph{#1}\mbox{}}
\fi
\makeatother


\providecommand{\tightlist}{%
  \setlength{\itemsep}{0pt}\setlength{\parskip}{0pt}}\usepackage{longtable,booktabs,array}
\usepackage{calc} % for calculating minipage widths
% Correct order of tables after \paragraph or \subparagraph
\usepackage{etoolbox}
\makeatletter
\patchcmd\longtable{\par}{\if@noskipsec\mbox{}\fi\par}{}{}
\makeatother
% Allow footnotes in longtable head/foot
\IfFileExists{footnotehyper.sty}{\usepackage{footnotehyper}}{\usepackage{footnote}}
\makesavenoteenv{longtable}
\usepackage{graphicx}
\makeatletter
\def\maxwidth{\ifdim\Gin@nat@width>\linewidth\linewidth\else\Gin@nat@width\fi}
\def\maxheight{\ifdim\Gin@nat@height>\textheight\textheight\else\Gin@nat@height\fi}
\makeatother
% Scale images if necessary, so that they will not overflow the page
% margins by default, and it is still possible to overwrite the defaults
% using explicit options in \includegraphics[width, height, ...]{}
\setkeys{Gin}{width=\maxwidth,height=\maxheight,keepaspectratio}
% Set default figure placement to htbp
\makeatletter
\def\fps@figure{htbp}
\makeatother
% definitions for citeproc citations
\NewDocumentCommand\citeproctext{}{}
\NewDocumentCommand\citeproc{mm}{%
  \begingroup\def\citeproctext{#2}\cite{#1}\endgroup}
\makeatletter
 % allow citations to break across lines
 \let\@cite@ofmt\@firstofone
 % avoid brackets around text for \cite:
 \def\@biblabel#1{}
 \def\@cite#1#2{{#1\if@tempswa , #2\fi}}
\makeatother
\newlength{\cslhangindent}
\setlength{\cslhangindent}{1.5em}
\newlength{\csllabelwidth}
\setlength{\csllabelwidth}{3em}
\newenvironment{CSLReferences}[2] % #1 hanging-indent, #2 entry-spacing
 {\begin{list}{}{%
  \setlength{\itemindent}{0pt}
  \setlength{\leftmargin}{0pt}
  \setlength{\parsep}{0pt}
  % turn on hanging indent if param 1 is 1
  \ifodd #1
   \setlength{\leftmargin}{\cslhangindent}
   \setlength{\itemindent}{-1\cslhangindent}
  \fi
  % set entry spacing
  \setlength{\itemsep}{#2\baselineskip}}}
 {\end{list}}
\usepackage{calc}
\newcommand{\CSLBlock}[1]{\hfill\break\parbox[t]{\linewidth}{\strut\ignorespaces#1\strut}}
\newcommand{\CSLLeftMargin}[1]{\parbox[t]{\csllabelwidth}{\strut#1\strut}}
\newcommand{\CSLRightInline}[1]{\parbox[t]{\linewidth - \csllabelwidth}{\strut#1\strut}}
\newcommand{\CSLIndent}[1]{\hspace{\cslhangindent}#1}

\KOMAoption{captions}{tableheading}
\makeatletter
\@ifpackageloaded{tcolorbox}{}{\usepackage[skins,breakable]{tcolorbox}}
\@ifpackageloaded{fontawesome5}{}{\usepackage{fontawesome5}}
\definecolor{quarto-callout-color}{HTML}{909090}
\definecolor{quarto-callout-note-color}{HTML}{0758E5}
\definecolor{quarto-callout-important-color}{HTML}{CC1914}
\definecolor{quarto-callout-warning-color}{HTML}{EB9113}
\definecolor{quarto-callout-tip-color}{HTML}{00A047}
\definecolor{quarto-callout-caution-color}{HTML}{FC5300}
\definecolor{quarto-callout-color-frame}{HTML}{acacac}
\definecolor{quarto-callout-note-color-frame}{HTML}{4582ec}
\definecolor{quarto-callout-important-color-frame}{HTML}{d9534f}
\definecolor{quarto-callout-warning-color-frame}{HTML}{f0ad4e}
\definecolor{quarto-callout-tip-color-frame}{HTML}{02b875}
\definecolor{quarto-callout-caution-color-frame}{HTML}{fd7e14}
\makeatother
\makeatletter
\@ifpackageloaded{bookmark}{}{\usepackage{bookmark}}
\makeatother
\makeatletter
\@ifpackageloaded{caption}{}{\usepackage{caption}}
\AtBeginDocument{%
\ifdefined\contentsname
  \renewcommand*\contentsname{Table of contents}
\else
  \newcommand\contentsname{Table of contents}
\fi
\ifdefined\listfigurename
  \renewcommand*\listfigurename{List of Figures}
\else
  \newcommand\listfigurename{List of Figures}
\fi
\ifdefined\listtablename
  \renewcommand*\listtablename{List of Tables}
\else
  \newcommand\listtablename{List of Tables}
\fi
\ifdefined\figurename
  \renewcommand*\figurename{Figure}
\else
  \newcommand\figurename{Figure}
\fi
\ifdefined\tablename
  \renewcommand*\tablename{Table}
\else
  \newcommand\tablename{Table}
\fi
}
\@ifpackageloaded{float}{}{\usepackage{float}}
\floatstyle{ruled}
\@ifundefined{c@chapter}{\newfloat{codelisting}{h}{lop}}{\newfloat{codelisting}{h}{lop}[chapter]}
\floatname{codelisting}{Listing}
\newcommand*\listoflistings{\listof{codelisting}{List of Listings}}
\makeatother
\makeatletter
\makeatother
\makeatletter
\@ifpackageloaded{caption}{}{\usepackage{caption}}
\@ifpackageloaded{subcaption}{}{\usepackage{subcaption}}
\makeatother

\ifLuaTeX
  \usepackage{selnolig}  % disable illegal ligatures
\fi
\usepackage{bookmark}

\IfFileExists{xurl.sty}{\usepackage{xurl}}{} % add URL line breaks if available
\urlstyle{same} % disable monospaced font for URLs
\hypersetup{
  pdftitle={Embrace the Majesty},
  pdfauthor={Subhojit Maji},
  colorlinks=true,
  linkcolor={blue},
  filecolor={Maroon},
  citecolor={Blue},
  urlcolor={Blue},
  pdfcreator={LaTeX via pandoc}}


\title{Embrace the Majesty}
\author{Subhojit Maji}
\date{2024-12-13}

\begin{document}
\maketitle


\bookmarksetup{startatroot}

\chapter*{Want to know Me ?}\label{want-to-know-me}
\addcontentsline{toc}{chapter}{Want to know Me ?}

\markboth{Want to know Me ?}{Want to know Me ?}

I am \textbf{SUBHOJIT MAJI}, a B.Math Undergraduate at
\href{https://www.isibang.ac.in/}{Indian Statistical Institute,
Bangalore Centre} . My Special Interests are in \textbf{Probability
Theory} but in no way limited to it .

I am currently a member of the \textbf{Cultural Committee} of
\href{https://www.isibang.ac.in/}{ISI Bangalore} for the session 2024-25
as well as a member of the \textbf{LIMIT} 2024-25
\href{https://www.limitisi.in/resources/team/design}{Design Team}. This
site is created by me for sharing some great problems and theoretical
stuffs with y'all.

Hope you all would enjoy browsing through it \emph{:)} !!

\bookmarksetup{startatroot}

\chapter*{Course Web Page}\label{course-web-page}
\addcontentsline{toc}{chapter}{Course Web Page}

\markboth{Course Web Page}{Course Web Page}

The Legendary Course Works at a Legendary Research Institute(ISI Bang) .

\hfill\break

\begin{tcolorbox}[enhanced jigsaw, breakable, rightrule=.15mm, opacityback=0, arc=.35mm, colframe=quarto-callout-important-color-frame, colback=white, toprule=.15mm, left=2mm, bottomrule=.15mm, leftrule=.75mm]

\vspace{-3mm}\textbf{Sem I @ isibang}\vspace{3mm}

These are some of the works, I did in my B.Math Journey.

\subsection*{Real Analysis}\label{real-analysis}
\addcontentsline{toc}{subsection}{Real Analysis}

I got some
\href{https://drive.google.com/file/d/15aQW5bT4oBRqTgWve05l_CL1jxj97ieQ/view?usp=sharing}{Real
Analysis} Notes for you all, by
\href{https://www.isibang.ac.in/~bhat/}{Prof.~BV Rajaram Bhatt}.

\subsection*{Fundamentals of Computer
Programming}\label{fundamentals-of-computer-programming}
\addcontentsline{toc}{subsection}{Fundamentals of Computer Programming}

We did this course under
\href{https://isi.irins.org/profile/13204}{Prof.~Jishnu Gupta Biswas}
Here are a few resorces :-

\begin{enumerate}
\def\labelenumi{\arabic{enumi}.}
\tightlist
\item
  OG Book for
  \href{https://drive.google.com/file/d/1JgIz5I723BBlrkGD4n0qE4n5U89LD3XY/view?usp=sharing}{C
  Programming}
\item
  My B.Math
  \href{https://www.isibang.ac.in/~adean/infsys/database/Bmath/FCP.html}{FCP
  course link}
\end{enumerate}

\subsection*{Probability Theory}\label{probability-theory}
\addcontentsline{toc}{subsection}{Probability Theory}

One of the favourite topics of my interest. I would like to share my
views on it(COMING SOON !!). This course is being taught by
\href{https://sites.google.com/view/mjoseph/home}{Prof.~Mathew Joseph}.

\subsection*{Elementary Number Theory}\label{elementary-number-theory}
\addcontentsline{toc}{subsection}{Elementary Number Theory}

Here are
\href{https://drive.google.com/file/d/1a-w8kH7kZud-1WypK9IrwDW7UoYUFmZf/view?usp=sharing}{Assignments}
that were given to us as a Homework. The course was taken by
\href{https://sites.google.com/view/ramdinmawia/home}{Prof.~Ramdin
Mawia}.

\subsection*{Linear Algebra}\label{linear-algebra}
\addcontentsline{toc}{subsection}{Linear Algebra}

This was without a doubt the best course in the 1st semester. I got to
learn a lot of new stuff and got myself acquainted with proof writing
and stuff like Resonance Articles. It was taken by
\href{https://www.isibang.ac.in/~statmath/homepage.html}{Prof.~Anita
Naolekar}. Here is a
\href{https://www.isibang.ac.in/~adean/infsys/database/Bmath/LAlg1.html}{link}
to the course page.

\subsection*{}\label{section}
\addcontentsline{toc}{subsection}{}

\(\textbf{Site Building in progress...}\)

\end{tcolorbox}

\bookmarksetup{startatroot}

\chapter*{Pro-Bable}\label{pro-bable}
\addcontentsline{toc}{chapter}{Pro-Bable}

\markboth{Pro-Bable}{Pro-Bable}

Ok! I know you would contradict with me but there is always a
probability that \emph{I am God}

\hfill\break

\begin{tcolorbox}[enhanced jigsaw, breakable, rightrule=.15mm, opacityback=0, toptitle=1mm, colframe=quarto-callout-note-color-frame, bottomrule=.15mm, toprule=.15mm, leftrule=.75mm, colbacktitle=quarto-callout-note-color!10!white, bottomtitle=1mm, coltitle=black, titlerule=0mm, colback=white, title=\textcolor{quarto-callout-note-color}{\faInfo}\hspace{0.5em}{Spread of Rumours}, left=2mm, arc=.35mm, opacitybacktitle=0.6]

In a small town of n people a person passes a titbit of information to
another person. A rumour is now launched with each recipient of the
information passing it on to a randomly chosen individual. What is the
probability that the rumour is told r times \(\textit{without}\)

\begin{enumerate}
\def\labelenumi{(\alph{enumi})}
\item
  returning to the originator,
\item
  being repeated to anyone.
\end{enumerate}

\begin{tcolorbox}[enhanced jigsaw, breakable, rightrule=.15mm, opacityback=0, toptitle=1mm, colframe=quarto-callout-tip-color-frame, bottomrule=.15mm, toprule=.15mm, leftrule=.75mm, colbacktitle=quarto-callout-tip-color!10!white, bottomtitle=1mm, coltitle=black, titlerule=0mm, colback=white, title=\textcolor{quarto-callout-tip-color}{\faLightbulb}\hspace{0.5em}{Generalisation}, left=2mm, arc=.35mm, opacitybacktitle=0.6]

Redo the calculations if each person tells the rumour to m randomly
selected people .

\end{tcolorbox}

\end{tcolorbox}

\begin{tcolorbox}[enhanced jigsaw, breakable, rightrule=.15mm, opacityback=0, toptitle=1mm, colframe=quarto-callout-note-color-frame, bottomrule=.15mm, toprule=.15mm, leftrule=.75mm, colbacktitle=quarto-callout-note-color!10!white, bottomtitle=1mm, coltitle=black, titlerule=0mm, colback=white, title=\textcolor{quarto-callout-note-color}{\faInfo}\hspace{0.5em}{Increasing Seuence of Events}, left=2mm, arc=.35mm, opacitybacktitle=0.6]

Suppose \(\set{A_n \vert n>1}\) is an increasing sequence of events such
that , \(A_n \subseteq A_{n+1}\) for each n . Let A =
\(\bigcup_{n=i}^\infty A_n\) . Show that \(P(A_n)\) \(\rightarrow P(A)\)
as n \(\rightarrow \infty\) .

\end{tcolorbox}

\begin{tcolorbox}[enhanced jigsaw, breakable, rightrule=.15mm, opacityback=0, toptitle=1mm, colframe=quarto-callout-note-color-frame, bottomrule=.15mm, toprule=.15mm, leftrule=.75mm, colbacktitle=quarto-callout-note-color!10!white, bottomtitle=1mm, coltitle=black, titlerule=0mm, colback=white, title=\textcolor{quarto-callout-note-color}{\faInfo}\hspace{0.5em}{Dice Game}, left=2mm, arc=.35mm, opacitybacktitle=0.6]

If three dice are thrown. What is the probability that one shows a 6
given that no two show the same face? Repeat for n dice where
\(2 < n < 6\)

\end{tcolorbox}

\begin{tcolorbox}[enhanced jigsaw, breakable, rightrule=.15mm, opacityback=0, toptitle=1mm, colframe=quarto-callout-note-color-frame, bottomrule=.15mm, toprule=.15mm, leftrule=.75mm, colbacktitle=quarto-callout-note-color!10!white, bottomtitle=1mm, coltitle=black, titlerule=0mm, colback=white, title=\textcolor{quarto-callout-note-color}{\faInfo}\hspace{0.5em}{How many couples will Survive ?}, left=2mm, arc=.35mm, opacitybacktitle=0.6]

Suppose there are \(2n\) individuals who constitute \(n\) married
couples at a given initial date. We wish to consider the survivors at
some given later date. Assume each individual has a probability \(p\) of
surviving until the later date (and possibly more!), independent of the
others. Let \(s\) denote the number of survivors at the later date and
\(c\) denote the number of surviving couples (in which both partners are
alive). The problem is to derive an expression for
\(\mathbb{E}(c \mid s)\) and show that it does not depend on \(p\). What
about \(\mathbb{E}(s \mid c)\)?

\end{tcolorbox}

\bookmarksetup{startatroot}

\chapter*{ALL-ZEBRAS}\label{all-zebras}
\addcontentsline{toc}{chapter}{ALL-ZEBRAS}

\markboth{ALL-ZEBRAS}{ALL-ZEBRAS}

Here are some Zebras to Hunt for !!

\hfill\break

\begin{tcolorbox}[enhanced jigsaw, breakable, rightrule=.15mm, opacityback=0, toptitle=1mm, colframe=quarto-callout-tip-color-frame, bottomrule=.15mm, toprule=.15mm, leftrule=.75mm, colbacktitle=quarto-callout-tip-color!10!white, bottomtitle=1mm, coltitle=black, titlerule=0mm, colback=white, title=\textcolor{quarto-callout-tip-color}{\faLightbulb}\hspace{0.5em}{The Most Common Determinant}, left=2mm, arc=.35mm, opacitybacktitle=0.6]

Determine the value of the determinant of

\begin{vmatrix}\gcd (1,1)&\gcd (1,2)&\cdots &\gcd (1,n)\\\gcd (2,1)&\gcd (2,2)&\cdots &\gcd (2,n)\\\vdots&\vdots&\ddots&\vdots\\\gcd (n,1)&\gcd (n,2)&\cdots &\gcd (n,n)\end{vmatrix}

\hfill\break

\end{tcolorbox}

\begin{tcolorbox}[enhanced jigsaw, breakable, rightrule=.15mm, opacityback=0, toptitle=1mm, colframe=quarto-callout-tip-color-frame, bottomrule=.15mm, toprule=.15mm, leftrule=.75mm, colbacktitle=quarto-callout-tip-color!10!white, bottomtitle=1mm, coltitle=black, titlerule=0mm, colback=white, title=\textcolor{quarto-callout-tip-color}{\faLightbulb}\hspace{0.5em}{Some Eigen Stuffs}, left=2mm, arc=.35mm, opacitybacktitle=0.6]

Distinct Eigen-Values correspond to linearly independent
Eigen-Vectors.(Prove it !!)

\end{tcolorbox}

\begin{tcolorbox}[enhanced jigsaw, breakable, rightrule=.15mm, opacityback=0, toptitle=1mm, colframe=quarto-callout-tip-color-frame, bottomrule=.15mm, toprule=.15mm, leftrule=.75mm, colbacktitle=quarto-callout-tip-color!10!white, bottomtitle=1mm, coltitle=black, titlerule=0mm, colback=white, title=\textcolor{quarto-callout-tip-color}{\faLightbulb}\hspace{0.5em}{Series is good for Health}, left=2mm, arc=.35mm, opacitybacktitle=0.6]

Let \(a_n = (\ln 3)^n \sum_{k=1}^{n} \frac{k^2}{k!(n-k)!}\) . Then the
sum of the series \(a_1 + a_2 + a_3 + ... \to \infty\) , is equal to?

\begin{tcolorbox}[enhanced jigsaw, breakable, rightrule=.15mm, opacityback=0, arc=.35mm, colframe=quarto-callout-color-frame, colback=white, toprule=.15mm, left=2mm, bottomrule=.15mm, leftrule=.75mm]

\vspace{-3mm}\textbf{Solution}\vspace{3mm}

\href{https://youtu.be/S56NGhteqCE?si=j_5-WG_R9wq7LGlb}{Video Solution}

\end{tcolorbox}

\end{tcolorbox}

\begin{tcolorbox}[enhanced jigsaw, breakable, rightrule=.15mm, opacityback=0, toptitle=1mm, colframe=quarto-callout-tip-color-frame, bottomrule=.15mm, toprule=.15mm, leftrule=.75mm, colbacktitle=quarto-callout-tip-color!10!white, bottomtitle=1mm, coltitle=black, titlerule=0mm, colback=white, title=\textcolor{quarto-callout-tip-color}{\faLightbulb}\hspace{0.5em}{Same-Same Yet Different!!}, left=2mm, arc=.35mm, opacitybacktitle=0.6]

A is a \(n \times m\) matrix and B is a \(m \times n\) matrix . If
\(|I_{n} - AB|\) is non-singular so is \(|I_{m} - BA|\)

\end{tcolorbox}

\begin{tcolorbox}[enhanced jigsaw, breakable, rightrule=.15mm, opacityback=0, toptitle=1mm, colframe=quarto-callout-tip-color-frame, bottomrule=.15mm, toprule=.15mm, leftrule=.75mm, colbacktitle=quarto-callout-tip-color!10!white, bottomtitle=1mm, coltitle=black, titlerule=0mm, colback=white, title=\textcolor{quarto-callout-tip-color}{\faLightbulb}\hspace{0.5em}{INMO 2024}, left=2mm, arc=.35mm, opacitybacktitle=0.6]

For each positive integer \(n \ge 3\), define \(A_n\) and \(B_n\) as
\(A_n = \sqrt{n^2 + 1} + \sqrt{n^2 + 3} + \cdots + \sqrt{n^2+2n-1}\) ,
\(B_n = \sqrt{n^2 + 2} + \sqrt{n^2 + 4} + \cdots + \sqrt{n^2 + 2n}.\)
Determine all positive integers \(n\ge 3\) for which
\(\lfloor A_n \rfloor = \lfloor B_n \rfloor\).

\begin{tcolorbox}[enhanced jigsaw, breakable, rightrule=.15mm, opacityback=0, toptitle=1mm, colframe=quarto-callout-note-color-frame, bottomrule=.15mm, toprule=.15mm, leftrule=.75mm, colbacktitle=quarto-callout-note-color!10!white, bottomtitle=1mm, coltitle=black, titlerule=0mm, colback=white, title={Note}, left=2mm, arc=.35mm, opacitybacktitle=0.6]

For any real number \(x\), \(\lfloor x\rfloor\) denotes the largest
integer \(N\le x\).

\end{tcolorbox}

\end{tcolorbox}

\bookmarksetup{startatroot}

\chapter*{Puzzles for our Young
Minds}\label{puzzles-for-our-young-minds}
\addcontentsline{toc}{chapter}{Puzzles for our Young Minds}

\markboth{Puzzles for our Young Minds}{Puzzles for our Young Minds}

Uff!! Here comes the relief from some routine course works.

\hfill\break

\begin{tcolorbox}[enhanced jigsaw, breakable, rightrule=.15mm, opacityback=0, toptitle=1mm, colframe=quarto-callout-note-color-frame, bottomrule=.15mm, toprule=.15mm, leftrule=.75mm, colbacktitle=quarto-callout-note-color!10!white, bottomtitle=1mm, coltitle=black, titlerule=0mm, colback=white, title=\textcolor{quarto-callout-note-color}{\faInfo}\hspace{0.5em}{Iterated Sine}, left=2mm, arc=.35mm, opacitybacktitle=0.6]

Consider the function \(f_n(x)\) defined for positive integers \(n\) and
real numbers \(x\) by \(f_n(x) = \sin(\sin(\sin(\cdots (\sin x)))),\)
with \(n\) applications of the sine function. Thus
\(f_2(x) = \sin(\sin x), \quad f_3(x) = \sin(\sin(\sin x)),\) and so on.

Let \(x \in (0, \pi/2)\) be chosen, and let the sequence
\(f_1(x), f_2(x), f_3(x), \dots, f_{100}(x),\) be computed. It is found
that for each fixed \(x\) and for all sufficiently large \(n\), we have
\(f_n(x) \approx \sqrt{\frac{3}{n}}.\) How may this phenomenon be
explained?

\begin{tcolorbox}[enhanced jigsaw, breakable, rightrule=.15mm, opacityback=0, toptitle=1mm, colframe=quarto-callout-tip-color-frame, bottomrule=.15mm, toprule=.15mm, leftrule=.75mm, colbacktitle=quarto-callout-tip-color!10!white, bottomtitle=1mm, coltitle=black, titlerule=0mm, colback=white, title=\textcolor{quarto-callout-tip-color}{\faLightbulb}\hspace{0.5em}{Solution}, left=2mm, arc=.35mm, opacitybacktitle=0.6]

\(\href{https://youtu.be/qN-dj94NHNY?si=zgsGB23e6bzrYGQl}{\textbf{Video Solution}}\)

\end{tcolorbox}

\end{tcolorbox}

\begin{tcolorbox}[enhanced jigsaw, breakable, rightrule=.15mm, opacityback=0, toptitle=1mm, colframe=quarto-callout-note-color-frame, bottomrule=.15mm, toprule=.15mm, leftrule=.75mm, colbacktitle=quarto-callout-note-color!10!white, bottomtitle=1mm, coltitle=black, titlerule=0mm, colback=white, title=\textcolor{quarto-callout-note-color}{\faInfo}\hspace{0.5em}{Maximums}, left=2mm, arc=.35mm, opacitybacktitle=0.6]

Assume the quartic polynomial \(x^4 - ax^3 + bx^2 - ax + d = 0\) has
four real roots namely, \(\frac{1}{2} \leq x_1, x_2, x_3, x_4 \leq 2\).
Find the maximum possible value of
\(\frac{(x_1 + x_2)(x_1 + x_3)x_4}{(x_4 + x_2)(x_4 + x_3)x_1}.\)

\begin{tcolorbox}[enhanced jigsaw, breakable, rightrule=.15mm, opacityback=0, toptitle=1mm, colframe=quarto-callout-tip-color-frame, bottomrule=.15mm, toprule=.15mm, leftrule=.75mm, colbacktitle=quarto-callout-tip-color!10!white, bottomtitle=1mm, coltitle=black, titlerule=0mm, colback=white, title=\textcolor{quarto-callout-tip-color}{\faLightbulb}\hspace{0.5em}{Solution}, left=2mm, arc=.35mm, opacitybacktitle=0.6]

\(\href{https://www.youtube.com/watch?v=o-mX3byaZ5k}{\textbf{Video Solution}}\)

\end{tcolorbox}

\end{tcolorbox}

\begin{tcolorbox}[enhanced jigsaw, breakable, rightrule=.15mm, opacityback=0, toptitle=1mm, colframe=quarto-callout-note-color-frame, bottomrule=.15mm, toprule=.15mm, leftrule=.75mm, colbacktitle=quarto-callout-note-color!10!white, bottomtitle=1mm, coltitle=black, titlerule=0mm, colback=white, title=\textcolor{quarto-callout-note-color}{\faInfo}\hspace{0.5em}{Polynomial Approximations}, left=2mm, arc=.35mm, opacitybacktitle=0.6]

Prove that :

\begin{quote}
The set of polynomial functions on \([a,b]\) is dense in \(C([a,b])\)
\end{quote}

\begin{tcolorbox}[enhanced jigsaw, breakable, rightrule=.15mm, opacityback=0, toptitle=1mm, colframe=quarto-callout-tip-color-frame, bottomrule=.15mm, toprule=.15mm, leftrule=.75mm, colbacktitle=quarto-callout-tip-color!10!white, bottomtitle=1mm, coltitle=black, titlerule=0mm, colback=white, title=\textcolor{quarto-callout-tip-color}{\faLightbulb}\hspace{0.5em}{Hints}, left=2mm, arc=.35mm, opacitybacktitle=0.6]

The following problem can be rephrased as follows

\begin{quote}
For every continuous function \(f(x)\) defined on \([a,b]\) , there is a
nice polynomial \(p(x)\) that approximates it ie,
\(\forall \epsilon > 0\) there exists p(x) such that
\(|f(x) - p(x)| < \epsilon \; \forall x \in [a,b]\) , these polynomials
are called as \(\textbf{Bernstein Polynomials}\)
\end{quote}

\begin{tcolorbox}[enhanced jigsaw, breakable, rightrule=.15mm, opacityback=0, arc=.35mm, colframe=quarto-callout-color-frame, colback=white, toprule=.15mm, left=2mm, bottomrule=.15mm, leftrule=.75mm]

\vspace{-3mm}\textbf{Some Facts}\vspace{3mm}

\begin{itemize}
\tightlist
\item
  As there is a bijection , show that it holds for \(C[(0,1)]\) and it
  holds true for \(C[(a,b)]\)
\item
  Let's denote the k-th Binomial Coefficient of \((1+x-x)^n\) as
  \(B_{n,k}\) = \(\binom{n}{k} x^k ({1-x})^{n-k}\) .
\end{itemize}

Then the n-th Bernstein Polynomial is \(P_{f,n}(x)\) =
\(\sum_{k=0}^{n} f\left(\frac{k}{n}\right)B_{n,k}\)

\end{tcolorbox}

\end{tcolorbox}

\end{tcolorbox}

\begin{tcolorbox}[enhanced jigsaw, breakable, rightrule=.15mm, opacityback=0, toptitle=1mm, colframe=quarto-callout-note-color-frame, bottomrule=.15mm, toprule=.15mm, leftrule=.75mm, colbacktitle=quarto-callout-note-color!10!white, bottomtitle=1mm, coltitle=black, titlerule=0mm, colback=white, title=\textcolor{quarto-callout-note-color}{\faInfo}\hspace{0.5em}{Smallest Set of integers}, left=2mm, arc=.35mm, opacitybacktitle=0.6]

Let \(S\) be the smallest set of positive integers such that

\begin{enumerate}
\def\labelenumi{\alph{enumi})}
\tightlist
\item
  \(2\) is in \(S,\)
\item
  \(n\) is in \(S\) whenever \(n^2\) is in \(S,\) and
\item
  \((n+5)^2\) is in \(S\) whenever \(n\) is in \(S.\)
\end{enumerate}

Which positive integers are not in \(S?\)

\begin{tcolorbox}[enhanced jigsaw, breakable, rightrule=.15mm, opacityback=0, toptitle=1mm, colframe=quarto-callout-tip-color-frame, bottomrule=.15mm, toprule=.15mm, leftrule=.75mm, colbacktitle=quarto-callout-tip-color!10!white, bottomtitle=1mm, coltitle=black, titlerule=0mm, colback=white, title=\textcolor{quarto-callout-tip-color}{\faLightbulb}\hspace{0.5em}{Text Solution}, left=2mm, arc=.35mm, opacitybacktitle=0.6]

The only positive integers not in \(S\) are \(1\) and the set of numbers
divisible by \(5.\)

A lemma to start with: if \(n\in S,\) then \(n+5k\in S\) for all
\(k\in\mathbb{N}\) We get then by the steps \(n\to (n+5)^2\to n+5,\)
repeated as many times as needed.

Based on that lemma, we make cases in the argument for each residue
class mod \(5.\) We see right away that if we don't already have a
multiple of \(5,\) we're never going to get one. Beyond that, to show
the claim stated above, we must show that \(2,3,4,\) and \(6\) must be
in \(S.\) We already have \(2\in S\) as given. We have the chain
\(2\to 49\to 54^2.\) Note that \(54^2\equiv 1\pmod{5},\) which gives us
all larger numbers that \(\equiv 1\pmod{5}.\) One such larger number is
\(2^{16}=256^2.\) That gives us \(256^2\to 256 \to 16\to 4.\) From \(4\)
we have \(4\to 9\to 3.\) From \(16\) we can get to \(16+5\cdot 4=36,\)
and then \(36\to 6.\) Hence we have all of \(2,3,4,\) and \(6\) in \(S\)
and \(S\) must be as claimed.

We do note that the only way to go from a larger number to a smaller
number is to go from \(n^2\) to \(n.\) But \(1^2=1,\) so there is no way
to get \(1\) from a larger number.

\end{tcolorbox}

\end{tcolorbox}

\bookmarksetup{startatroot}

\chapter*{My-Journey}\label{my-journey}
\addcontentsline{toc}{chapter}{My-Journey}

\markboth{My-Journey}{My-Journey}

A brief history of me, in all its chaotic glory.

\hfill\break

\begin{tcolorbox}[enhanced jigsaw, breakable, rightrule=.15mm, opacityback=0, arc=.35mm, colframe=quarto-callout-note-color-frame, colback=white, toprule=.15mm, left=2mm, bottomrule=.15mm, leftrule=.75mm]

\vspace{-3mm}\textbf{Let it be Hidden :relaxed:}\vspace{3mm}

\begin{itemize}
\tightlist
\item
  Qualified for RMO twice(in the years 2020 and 2023).
\item
  Qualified JEE Mains in 2024(With 99.57 percentile).
\item
  Qualified JEE Advanced.
\item
  Qualified WBJEE in 2024(Rank 291).
\item
  Participated at OPhO(Online Physics Olympiad by Jaan Kalda).
\item
  Qualified ISI B.Math Entrance Exam.
\item
  Bronze Medalist at DIGO(Conducted by AoPS).
\item
  Recieved the Centre Topper Certificates for NSEP, NSEC and NSEA.
\item
  CBSE Boards Class X - 95\% and Class XII - 93\% .
\item
  Pursuing my UDGRP(Undergraduate Directed Group Reading Project) 2024
  (NOT COMPLETED) \ldots{}
\item
  Got the Offer Letter from IISc Bangalore for Maths and Computing and
  B.S Mathematics through JEE.
\end{itemize}

\end{tcolorbox}

\part{UDGRP's}

\chapter*{Winter UDGRP 24}\label{winter-udgrp-24}
\addcontentsline{toc}{chapter}{Winter UDGRP 24}

\markboth{Winter UDGRP 24}{Winter UDGRP 24}

The secret handshake of information, passed down through generations.

\hfill\break

\begin{tcolorbox}[enhanced jigsaw, breakable, rightrule=.15mm, opacityback=0, toptitle=1mm, colframe=quarto-callout-note-color-frame, bottomrule=.15mm, toprule=.15mm, leftrule=.75mm, colbacktitle=quarto-callout-note-color!10!white, bottomtitle=1mm, coltitle=black, titlerule=0mm, colback=white, title=\textcolor{quarto-callout-note-color}{\faInfo}\hspace{0.5em}{Enumerative Combinatorics (By
\href{https://hrishik-koley.github.io}{Hrishik Koley})}, left=2mm, arc=.35mm, opacitybacktitle=0.6]

\begin{quote}
Here we were introduced to :-

\begin{itemize}
\tightlist
\item
  A bit of Group Theory(Mostly Geometric Group Theory)
\item
  Generating Functions
\item
  Graph Theory

  \begin{itemize}
  \tightlist
  \item
    Planar Graphs
  \item
    Perfect Graphs
  \item
    Ramsey's Number and Theorem
  \end{itemize}
\end{itemize}
\end{quote}

Resources and detailed Overview of the
\href{https://hrishik-koley.github.io/musings/enumerative_combinatorics/}{Course}

\end{tcolorbox}

\begin{tcolorbox}[enhanced jigsaw, breakable, rightrule=.15mm, opacityback=0, toptitle=1mm, colframe=quarto-callout-note-color-frame, bottomrule=.15mm, toprule=.15mm, leftrule=.75mm, colbacktitle=quarto-callout-note-color!10!white, bottomtitle=1mm, coltitle=black, titlerule=0mm, colback=white, title=\textcolor{quarto-callout-note-color}{\faInfo}\hspace{0.5em}{Introduction to Dynamical Systems(By
\href{https://paulpseudoman.github.io/}{Aritrabha Majumdar})}, left=2mm, arc=.35mm, opacitybacktitle=0.6]

It was a basic introductory course where we learnt about the following
stuffs -

\begin{itemize}
\tightlist
\item
  Some Basic Linear Algebra
\item
  Stability of Ordinary Differential Equations

  \begin{itemize}
  \tightlist
  \item
    Linear Systems
  \item
    Non-Linear Systems(Local Theory)
  \end{itemize}
\end{itemize}

Also at the middle of the course I made a presentation on the
\href{https://drive.google.com/file/d/1bnqrIGg4x6yNU18mimoggvLsltV1oL1Q/view?usp=sharing}{Spectral
Theorem} .

Here are the notes
\href{https://paulpseudoman.github.io/DSnotes.pdf}{link} to the course.

\end{tcolorbox}

\bookmarksetup{startatroot}

\chapter*{Summary}\label{summary}
\addcontentsline{toc}{chapter}{Summary}

\markboth{Summary}{Summary}

`There is always a probability of winning in life , no matter which
stage you are at. Just believe in yourself and make the best choices.'

\bookmarksetup{startatroot}

\chapter*{References}\label{references}
\addcontentsline{toc}{chapter}{References}

\markboth{References}{References}

\begin{enumerate}
\def\labelenumi{\arabic{enumi}.}
\tightlist
\item
  A Basic Introduction to
  \href{https://www.ias.ac.in/article/fulltext/reso/001/02/0055-0068}{Randomness}
  by Rajeeva L Karandikar .
\end{enumerate}

\phantomsection\label{refs}
\begin{CSLReferences}{0}{1}
\end{CSLReferences}




\end{document}
